
\chapter{Opis założeń projektu}
\label{chap:zalozenia}

Głównym celem projektu jest \textbf{stworzenie w pełni funkcjonalnej aplikacji desktopowej, która realizuje procesy sprzedaży od strony klienta oraz zarządzania sklepem od strony administratora}. Aplikacja została zaprojektowana jako kompletne rozwiązanie symulujące działanie komercyjnego systemu sprzedaży, które z jednej strony oferuje intuicyjny interfejs dla klienta, umożliwiający przeglądanie produktów i realizację zakupów, a z drugiej strony wyposażone jest w rozbudowany panel administracyjny dający pełną kontrolę nad kluczowymi aspektami funkcjonowania sklepu.

\section{Wymagania Funkcjonalne}
Aplikacja musi realizować następujące kluczowe funkcje, które zostały zdefiniowane w celu zapewnienia pełnej obsługi procesów sprzedażowych i administracyjnych.

\textbf{Uwierzytelnianie użytkowników.} System musi weryfikować tożsamość użytkowników na podstawie loginu i hasła, precyzyjnie rozróżniając role administratora i klienta. 
\textbf{Przeglądanie produktów.} Wszyscy użytkownicy, zarówno niezalogowani, jak i zalogowani klienci, muszą mieć zapewnioną możliwość swobodnego przeglądania katalogu dostępnych produktów. 
\textbf{Zarządzanie koszykiem.} Klienci muszą mieć możliwość dodawania produktów do koszyka, przeglądania jego zawartości oraz łatwego usuwania z niego wybranych pozycji. 
\textbf{Składanie zamówień.} System musi umożliwiać finalizację procesu zakupu poprzez dedykowany formularz, w którym klient podaje swoje dane niezbędne do pomyślnej realizacji transakcji. 
\textbf{Panel Administracyjny.} Administrator po zalogowaniu otrzymuje dostęp do panelu zarządczego, który umożliwia kompleksowe zarządzanie asortymentem (dodawanie, edycja i usuwanie produktów) , kontrolowanie stanu magazynowego, przeglądanie i zarządzanie danymi zarejestrowanych klientów oraz monitorowanie wszystkich złożonych zamówień i transakcji. 
\section{Wymagania Niefunkcjonalne}
Aplikacja musi spełniać kluczowe wymagania niefunkcjonalne, aby zapewnić wysoką jakość działania. Po pierwsze, \textbf{wydajność}, gwarantująca, że aplikacja będzie działać płynnie, a czas odpowiedzi na akcje użytkownika nie przekroczy akceptowalnych norm. Po drugie, \textbf{niezawodność}, oznaczająca odporność systemu na błędy (w szczególności na problemy z połączeniem z bazą danych) i zdolność do klarownego informowania użytkownika o ewentualnych niepowodzeniach operacji. Po trzecie, \textbf{użyteczność}, dzięki której interfejs graficzny jest intuicyjny i prosty w obsłudze zarówno dla klienta, jak i administratora. Na koniec, \textbf{przenośność}, wynikająca z zastosowania języka Java, pozwala na uruchomienie aplikacji na dowolnym systemie operacyjnym z zainstalowaną maszyną wirtualną Javy (JVM).
