\chapter{Wizja i cele projektu}
\label{chap:wizja}
Celem projektu było stworzenie aplikacji desktopowej, która kompleksowo symuluje działanie sklepu. Wizja zakładała budowę narzędzia, które z jednej strony zapewnia klientom (zarówno detalicznym, jak i hurtowym) \textbf{prostą i wygodną ścieżkę zakupową}, a z drugiej daje administratorowi \textbf{rozbudowane centrum dowodzenia} do zarządzania całym zapleczem sklepu – od produktów i stanów magazynowych, po klientów i historię transakcji. Aplikacja ma stanowić w pełni funkcjonalny ekosystem sprzedażowy.

\section{Kluczowe możliwości aplikacji}
Aby zrealizować postawione cele, aplikacja musi oferować szereg przemyślanych funkcji, które razem tworzą spójne doświadczenie dla każdego użytkownika.
Każdy, nawet bez logowania, będzie mógł swobodnie przeglądać \textbf{katalog produktów}. Gdy użytkownik zdecyduje się na zakupy, system umożliwi mu założenie konta i zalogowanie się lub kontynuację bez zakładania konta. Proces logowania jest kluczowy, ponieważ na jego podstawie aplikacja rozpozna, czy ma do czynienia z klientem, czy z administratorem, i dostosuje dostępne opcje.
Dla \textbf{klienta} przewidziano intuicyjną ścieżkę zakupową: od dodawania produktów do \textbf{wirtualnego koszyka}, przez jego modyfikację, aż po sfinalizowanie transakcji w prostym formularzu zamówienia.
Dla \textbf{administratora}, po zalogowaniu, otworzy się dostęp do rozbudowanego \textbf{panelu zarządczego}. Będzie to jego centrum dowodzenia, w którym zyska pełną kontrolę nad asortymentem – będzie mógł dodawać nowe produkty, edytować istniejące i zarządzać stanami magazynowymi. Panel pozwoli również na przeglądanie listy zarejestrowanych klientów oraz monitorowanie wszystkich złożonych zamówień.

\section{Jakość i doświadczenie użytkownika}
Poza samymi funkcjami, fundamentalne znaczenie ma to, jak aplikacja będzie działać. Założeniem jest, aby korzystanie z niej było efektywne i bezproblemowe.
Priorytetem jest, aby aplikacja była \textbf{prosta w obsłudze i responsywna}. Zarówno klient przeglądający ofertę, jak i administrator wprowadzający nowy produkt, powinni czuć, że program działa \textbf{płynnie i intuicyjnie}. Każda akcja musi wywoływać natychmiastową, przewidywalną reakcję systemu, bez irytujących opóźnień.
Równie ważna jest \textbf{stabilność i niezawodność}. Aplikacja musi być przygotowana na nieprzewidziane sytuacje, takie jak chwilowe problemy z dostępem do bazy danych. W takim przypadku system nie może się zawiesić, lecz powinien w \textbf{zrozumiały dla użytkownika sposób poinformować go o problemie} i pozwolić na kontynuację pracy, gdy tylko będzie to możliwe.
Na koniec, dzięki zastosowaniu technologii Java, aplikacja będzie \textbf{uniwersalna}.