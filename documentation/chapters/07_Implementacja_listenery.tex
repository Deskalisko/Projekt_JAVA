\chapter{Implementacja i Kluczowe Fragmenty Kodu}
\section{Wybrane Metody Usprawniające Pracę}

\begin{minipage}{\linewidth}
\subsection{Metoda `createStyledButton` - Ujednolicenie Wyglądu Przycisków}
Metoda `createStyledButton` jest przykładem dobrej praktyki programistycznej, pozwalającej na tworzenie spójnych stylistycznie przycisków w interfejsie użytkownika. Dzięki niej unika się powielania kodu odpowiedzialnego za ustawianie czcionki, kolorów, ramek i innych właściwości wizualnych dla każdego przycisku z osobna. Znacząco przyspieszyło to budowanie interfejsu graficznego. 

\lstinputlisting[style=javaStyle, language=Java, caption=Metoda createStyledButton, label=lst:createStyledButton]{src/CreateStyledButton.java}
\end{minipage}

\vspace{1cm}

\begin{minipage}{\linewidth}
\subsection{Metoda `getConnection` - Efektywne Zarządzanie Połączeniami z Bazą Danych}
Zarządzanie połączeniami z bazą danych jest kluczowe dla wydajności i niezawodności aplikacji.  Metoda `getConnection` z klasy `DatabaseConnection` zapewnia scentralizowany i prosty sposób na uzyskanie aktywnego połączenia z bazą danych MySQL.  Jej wykorzystanie znacznie uprościło operacje na danych, minimalizując ryzyko błędów związanych z otwieraniem i zamykaniem połączeń. 

\lstinputlisting[style=javaStyle, language=Java, caption=Metoda getConnection, label=lst:getConnection]{src/DatabaseConnection.java}
\end{minipage}

\section{Inne Istotne Fragmenty Kodu}

\begin{minipage}{\linewidth}
\subsection{Integracja i Użycie Kalendarza \texttt{JDateChooser}}
W celu ułatwienia wyboru daty, komponent \texttt{JDateChooser} został zintegrowany z interfejsem użytkownika. Jest to szczególnie przydatne **w panelu zarządzania transakcjami administratora, gdzie umożliwia precyzyjne filtrowanie i wyszukiwanie transakcji według daty**. Komponent ten eliminuje konieczność ręcznego wprowadzania daty i ryzyko błędów formatowania, zapewniając jednocześnie spójny wygląd i format daty w całej aplikacji, a także w innych miejscach, gdzie wybór daty jest istotny. 
Finalny wygląd panelu z zaimplementowanym komponentem kalendarza przedstawiono na \figurename~\ref{fig:management_transaction}.

\lstinputlisting[style=javaStyle, language=Java, caption=Metoda createDateChooser, label=lst:createDateChooser]{src/CreateDateChooser.java}

\end{minipage}
