\chapter{Implementacja i Kluczowe Fragmenty Kodu}
\section{Wybrane Metody Usprawniające Pracę}

\begin{minipage}{\linewidth}
\subsection{Metoda \texttt{createStyledButton} - Ujednolicenie Wyglądu Przycisków}
Metoda `createStyledButton` jest przykładem dobrej praktyki programistycznej, pozwalającej na tworzenie spójnych stylistycznie przycisków w interfejsie użytkownika. Dzięki niej unika się powielania kodu odpowiedzialnego za ustawianie czcionki, kolorów, ramek i innych właściwości wizualnych dla każdego przycisku z osobna. Znacząco przyspieszyło to budowanie interfejsu graficznego, co widać na przykładzie przedstawionym w \listingname~\ref{lst:createStyledButton}. 

\lstinputlisting[style=javaStyle, language=Java, caption=Metoda createStyledButton, label=lst:createStyledButton]{src/CreateStyledButton.java}
\end{minipage}

\clearpage

\begin{minipage}{\linewidth}
\subsection{Metoda \texttt{getConnection} - Efektywne Zarządzanie Połączeniami z Bazą Danych}
Zarządzanie połączeniami z bazą danych jest kluczowe dla wydajności i niezawodności aplikacji. Metoda `getConnection` z klasy `DatabaseConnection` (\listingname~\ref{lst:getConnection}) zapewnia scentralizowany i prosty sposób na uzyskanie aktywnego połączenia z bazą danych MySQL. Jej wykorzystanie znacznie uprościło operacje na danych, minimalizując ryzyko błędów związanych z otwieraniem i zamykaniem połączeń. 

\lstinputlisting[style=javaStyle, language=Java, caption=Metoda getConnection, label=lst:getConnection]{src/DatabaseConnection.java}
\end{minipage}

\section{Inne Istotne Fragmenty Kodu}

\begin{minipage}{\linewidth}
\subsection{Integracja i Użycie Kalendarza \texttt{LGoodDatePicker}}
Aby ułatwić użytkownikowi wybór dat, w projekcie zintegrowano zewnętrzną bibliotekę \texttt{LGoodDatePicker} \cite{LGoodDatePicker}. Komponent ten jest kluczowy w panelu zarządzania transakcjami, gdzie pozwala administratorowi na intuicyjne filtrowanie wyników według zadanego okresu. Jego konfigurację w projekcie przedstawia \listingname~\ref{lst:createDatePicker}.

Zastosowanie gotowego kalendarza wyeliminowało konieczność ręcznego wprowadzania dat i ryzyko błędów formatowania, zapewniając spójny wygląd i wygodę obsługi w całej aplikacji. Finalny wygląd panelu z zaimplementowanym kalendarzem przedstawiono na \figurename~\ref{fig:management_transaction}.

\lstinputlisting[style=javaStyle, language=Java, caption=Metoda tworząca i konfigurująca komponent DatePicker., label=lst:createDatePicker]{src/CreateDateChooser.java}

\end{minipage}