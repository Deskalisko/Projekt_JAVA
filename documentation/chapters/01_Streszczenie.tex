\chapter{Streszczenie}
\label{chap:nowe_wprowadzenie}

\section{Streszczenie w języku polskim}
Głównym celem projektu było stworzenie desktopowego systemu do zarządzania sklepem. Aplikacja składa się z dwóch głównych modułów: interfejsu dla klienta (z podziałem na sprzedaż detaliczną i hurtową) oraz panelu dla administratora. System został napisany w języku \textbf{Java}, jego interfejs graficzny powstał przy użyciu biblioteki \textbf{Swing}, a za przechowywanie danych odpowiada baza \textbf{MySQL}. Zastosowanie wzorca projektowego \textbf{DAO} pozwoliło na oddzielenie logiki biznesowej od operacji na bazie danych, co ułatwia utrzymanie i rozwój kodu. Klient może przeglądać produkty i finalizować zakupy, a administrator zarządza asortymentem, klientami, transakcjami i stanami magazynowymi.

\section{Summary in English}
The main objective of this project was to create a desktop store management system. The application consists of two main modules: a customer interface (with separate retail and wholesale functionalities) and an administrator panel. The system was developed in \textbf{Java}, with its graphical user interface built using the \textbf{Swing} library, and a \textbf{MySQL} database for data storage. The use of the \textbf{DAO} design pattern allowed for the separation of business logic from database operations, simplifying code maintenance and scalability. The customer can browse products and complete purchases, while the administrator manages inventory, customers, transactions, and stock levels.